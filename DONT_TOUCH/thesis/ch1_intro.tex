\chapter{Introduction}
\label{c:intro}

Multiple access techniques are critical topics in cellular mobile communications. In most of the multiple access systems, users are allowed to transmit signals simultaneously through different types of user-specific signatures. However, the number of users that can be served in a multiple access system is usually limited due to the constraint of user-specific signatures in the allocated resources.  Compared to orthogonal multiple access schemes, non-orthogonal multiple access (NOMA)\cite{shahab2019grant} that allows the overlapping of multiple users over a radio resource can make the resource allocation more flexible.  In \cite{shahab2019grant}, a survey paper, briefly classified NOMA into different types such as the scrambling based Power Domain (PD)
-NOMA, the spreading based SCMA, the interleaving based IDMA.  In \cite{bh18}, a novel multiple access techniques called gain-division multiple access (GDMA) was investigated.  We can define GDMA as a gain-based NOMA system.

Different from PD-NOMA that multiple users are superposed according to the specific power domain, the gain-division multiple access (GDMA) utilizes the different combinations of amplitudes and phase shifts in channel coefficients caused by the independent fading channels. Hence, $U$ users are allowed to share the same transmit resource with multi-level detection in the GDMA system. The detection scheme of GDMA originated from the concept of physical layer network coding (PLNC), which coordinates transmissions among nodes, and packet collisions in traditional wireless networks can be eliminated \cite{plnc06}. 

In forward error coding GDMA system, there are two types of detection principles: separate channel decoding (SCD) and joint channel decoding (JCD). GDMA-SCD is a straightforward method that separates individual information from superimposed signal information, and there is no exchange of information between collided users. GDMA-JCD is a more sophisticated method that employs all information of superimposed signals to decode. In \cite{yt19}, a joint channel decoding and physical-layer network coding (G-JCNC)\cite{gjcnc10} was applied for the GDMA LDPC coded system. However, there is no effective JCD for polar code. We introduced JCD decoders for a polar coded system, which can effectively improve the error performances. 

Since the multiuser detection technique (MUD) in the GDMA system allows $U$ users to share the same resource, each of the received symbols consisting of the signals transmitted from $U$ users can be seen as a superimposed signal with $2^{mU}$ levels where $m$ is the modulation order. A blind channel estimation scheme\cite{yt19} was proposed for achieving spectral efficiency since no additional pilot signal is needed for channel estimation. Firstly, the system classifies the received symbols into $2^{mU}$ groups by applying a clustering algorithms\cite{lbg80}\cite{kmpp07}\cite{em77}, where the symbols corresponding to the same superimposed level are grouped together, the estimates of the levels can be attained through the average of the symbols in each group. Secondly, the system utilizes the geometrical configuration of clustering estimated $2^{mU}$ levels to derive an estimated $U$ channel gains. Thirdly, a phase ambiguity algorithm is used to remove the phase ambiguity. 
  
In this thesis, we proposed a general adaptive clustering algorithm that can reduce the complexity and achieve Mean Square Error (MSE) for channel estimation closer to the Cramer Rao (CR) bound as compared to the algorithm used in \cite{yt19}.  In \cite{yt19}, the algorithm for the first and second stages can only be applied for a small amount of users $U$ and for modulation with small order $m$. With this observation, we also propose an algorithm for deriving channel gains by recursive operations. This algorithm can be applied to any number $U$ of users and any constellation size $m$.

Recently, random access (RA) protocols have acquired a lot of attention, not only from satellite communication but also from researchers active in fields such as Internet of Things (IoT) and machine-to-machine communications. Random access is a multiuser communication system in which users transmit packets over a shared channel, which is random. When there is more than one packets in the time duration, these packets collide, which results in transmission failure. Based on the conventional RA scheme (ALOHA system), several random access protocols were developed.  Most of them used more channel resources to avoid collisions.  To tackle the condition for which there are many device connections and few channel resources, we introduce a novel RA scheme which resolves the collisions without additional channel resources by applying the GDMA concept. The scenarios of slotted, un-slotted, grant-based, and grant-free for the RA-GMDA system are investigated. The preamble signals are used for synchronization and channel estimation, where the preamble based channel estimation can be integrated into the clustering based channel estimation. 


The organization of this thesis is as follows. In Chapter~\ref{c:gdma}, basic concepts of the GDMA technique are reviewed ,and the system models are also illustrated. The modified cluster-based channel estimation for the GDMA system is proposed in Chapter~\ref{c:cluster_ce} including the clustering algorithms and the schemes to resolve the phase ambiguity problem. Moreover, the GDMA based random access scheme is proposed in Chapter~\ref{c:rach_gdma}. Concluding remarks and future works are presented in Chapter~\ref{c:remarks}.
