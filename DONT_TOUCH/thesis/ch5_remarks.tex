\chapter{Conclusion and Future Works}
\label{c:remarks}


In Chapter.\ref{c:gdma}, we reviewed the GDMA concept, which exploited the distinct channels to separate the signals transmitted from multiple users over independent fading channels and introduced a general GDMA detector in BPSK, QPSK, and PAM for implementing practically.   We also introduced a Polar-coded joint channel decoder that achieved better BLER performance, which can also be applied to the physical network coding area. However, the path metrics operation of the J-SCL decoder will grow enormously. In future work, the metric reduction is indeed.

In Chapter.\ref{c:cluster_ce}, The proposed cluster-based scheme for blind channel estimation can be applied to the transmissions over scalar channels, and high spectral efficiency can be obtained since no additional pilot signal is needed for channel estimation. We further proposed a general clustering algorithm and deriving channel gain algorithm for cluster-based channel estimation, which can achieve CRLB for any user $U$. However, the operation of deriving channel gain is also increased exponentially. For practical implantation, the algorithm reduction is also indeed.

In Chapter.\ref{c:rach_gdma}, we utensil the GDMA concept to detect collision RA packets in an OFDM system, which does not require additional user-specific resources to avoid a collision. The main research is unslotted RA-OFDM-GDMA, which is better than a blindly random access (ALOHA) system. However, the theory presentations for throughput and ISI noise are still weak. In future work, the theory derivation is indeed, and we can further apply it to the Doppler channel system for discussing the channel estimation effect.  

