\begin{abstractEN}

Gain division multiple access (GDMA) is a recently proposed multiple access technique that allows multiple users to share the same resource by exploiting distinct channel coefficients associated with distinct users. In this thesis, we conduct an extended study of GDMA and also its applications to random access.

We propose a general formula for expressing the superimposed signal of all users and the log-likelihood ratio of each user bit.  As to the polar coded GDMA, we propose a joint channel decoder that achieved better BLER performance as compared to the decoding using separate decoders.

In the literature, blind estimation schemes based on the clustering algorithm for the GDMA system were proposed.  In this thesis, we proposed improved methods as that the channel estimation can be closer to the Cramer Rao bound.   In the literature, blind estimation schemes based on the clustering algorithm allow only a small number of users and modulation with small constellation sizes.  We propose a method that can be applied to a random user number and a random constellation. 

\

In the random access techniques, uncoordinated users transmit packets randomly. We introduced a random access system using GDMA.  Zadoff-Chu sequence is used to assist the acquisition of synchronization and channel estimation. The resultant throughput is superior to the conventional random access system ,which fails upon collision without using multiple access.  


\

Key words: Multiple access, fading channel, channel estimation, blind estimation, clustering algorithm, Polar codes, Joint channel  decoder, Radom access, Zadoff-Chu sequence, Synchronization.

\end{abstractEN}
